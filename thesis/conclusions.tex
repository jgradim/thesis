\chapter{Conclusion}\label{chap:conclusion}

\section{Conclusions}\label{chap:conclusions}

\textbf{NOTE: these are only very preliminar notes on conclusions}

Despite being built with a MVC architecture, the Ruby On Rails framework is, using the right approach, capable of working with some architectural and design patterns not obviously connected with MVC and AR.

Although the majority of the work present in this thesis in concerned with increasing the variability of software systems, it is possible to conclude that the application of the appropriate design patterns is able to increase not only the variability and configurability needs of a specific part of an application, but also its performance. This should not be taken as an universal truth, as it depends on a series of factors that may not be present in all implementations.

\section{Further Developments}\label{sec:further_developments}

%\section{Future works} ???

%Adaptive Object-Model architectures provide the best framework for building adaptable systems that are passable of modification by the end-users (which assumes no compilation or deployment processes). A lot of thought and research has gone into the best practices for the complete implementation of these types of systems, from model creation, maintenance and persistence to GUI generation.

%There has been very few work regarding true adaptive systems on the web. However, a considerable amount of research has been made regarding the best mechanisms for end-user website customization, which can be used to create web applications based on AOM architectures. 
