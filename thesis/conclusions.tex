\chapter{Conclusion}\label{chap:conclusion}

The problem described in chapter \ref{chap:problem_statement} has been solved according to the solution presented in chapter \ref{chap:approach_results} and the results achieved from this approach have been presented in the same chapter. In this last chapter, Section \ref{sec:conclusions}, some last remarks will be made about this study, as well as a summary of the main achievements and results. In Section \ref{sec:further_developments} the further improvements to the project will be presented and in Section \ref{sec:future_works} the future works will be described.

\section{Conclusions}\label{sec:conclusions}

Despite being built with a MVC architecture, the Ruby On Rails framework is, using the right approach, capable of working with some architectural and design patterns not obviously connected with MVC and AR. The application of these patterns is capable of increasing the level of variability present in a common Rails application, and a harmonious integration with the Rails 2.3.x infrastructure --- especially the \textsc{ActiveRecord} engine --- is possible and works elegantly.

Although the majority of the work present in this dissertation in concerned with increasing the variability of software systems, it is possible to conclude that the application of the appropriate design patterns is able to increase not only the variability and configurability needs of a specific part of an application, but also its performance. However, this should not be taken as an universal truth, as it depends on a series of factors that may not be present in all implementations, such as a previous innapropriate design.

\section{Further Developments}\label{sec:further_developments}

The main improvements and additions that can be made to the final work presented are divided into 2 groups:

\begin{itemize}
 \item Implement everything!
 \item Keep the accountabilities in memory
\end{itemize}

\section{Future Works}\label{sec:future_works}

In this section it will be discussed the projects that can be derived from this study as well as the applications and domains that can be dealt using the same methodological approach.

 * Test integration of these pattern with Rails 3

%Adaptive Object-Model architectures provide the best framework for building adaptable systems that are passable of modification by the end-users (which assumes no compilation or deployment processes). A lot of thought and research has gone into the best practices for the complete implementation of these types of systems, from model creation, maintenance and persistence to GUI generation.

%There has been very few work regarding true adaptive systems on the web. However, a considerable amount of research has been made regarding the best mechanisms for end-user website customization, which can be used to create web applications based on AOM architectures. 
