\chapter{Problem Statement}\label{chap:problem_statement}

While Ruby on Rails is built to maximize the developers productivity, it does so by providing a comprehensive set of tools that perform most of the work. The RoR framework is also designed for easy system evolution through \emph{migrations} --- which allow developers to easily add, remove and modify tables and table rows, while minimizing the effort of maintaining application consistency. While this is a solid, proven approach to improve an application's variability over time, it is often less flexible than one might wish, often leading to complicated data migration tasks which may involve modifying production data while ensuring consistency --- this can pose as a problem when the ammount of data is extensive and highly variable in nature. As such, the use of a static database schema coupled with migrations may not be entirely desireable for applications highly variable in nature.

As presented on Chapter~\ref{chap:sota}, the usage and application of adequate design patterns is capable of mitigating some of these issues, improving both an application's variability and maintainability.

\section{Design Patterns}\label{sec:design_patterns}

\section{Architectural Patterns}\label{sec:architectural_patterns}

