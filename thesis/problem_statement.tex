\chapter{Problem Statement}\label{chap:problem_statement}

While Ruby on Rails is built to maximize the developers productivity, it does so by providing a comprehensive set of tools that perform most of the work. The RoR framework is also designed for easy system evolution through \emph{migrations} --- which allow developers to easily add, remove and modify tables and table rows, while minimizing the effort of maintaining application consistency. While this is a solid, proven approach to improve an application's variability over time, it is often less flexible than one might wish, often leading to complicated data migration tasks which may involve modifying production data while ensuring consistency --- this can pose as a problem when the ammount of data is extensive and highly variable in nature. As such, the use of a static database schema coupled with migrations may not be entirely desireable for applications highly variable in nature.

As presented on Chapter~\ref{chap:sota}, the usage and application of adequate design patterns is capable of mitigating some of these issues, improving both an application's variability and maintainability.

The main concern of this thesis is how these architectural and design patterns can be effectively applied to a somewhat (architecturaly speaking) restrictive framework, and how these techniques can be combined with a schema-oriented MVC architecture used by the Ruby on Rails Framework.

FIXME: It is also concerned with how can the variability of such systems increase and how effective they are in terms of development and performance.

\section{Architectural Patterns}\label{sec:architectural_patterns}

The application of architectural patterns is outside the scope of this project, mainly because the project which will serve as a base to the case studies herein described is tied to the Ruby on Rails framework and its implementation of the MVC architecture for software systems. Trying to tackle the problem by changing the underlying architecture would mean that the application would mostly have to be rebuilt from scratch, with the team having to stop development to rewrite the application and learning a whole new set of skills, which is simply not feasible. FIXME: Also, financial issues with having the whole team (it's a small team) training instead of developing and enhancing the existing product. % Many companies are tied to obsolete software for the same reasons

\section{Design Patterns}\label{sec:design_patterns}

As the modification of the underlying architecture has been discarded for the reasons stated in \ref{sec:architectural_patterns}, the next step is to modify only \emph{parts} of the application design, focusing on a number of different areas --- or hotspots --- chosen for their unique requirements in terms of variability.

Which AOM-related design patterns should be applied to solve certain problems, and why

