\chapter{Problem Statement}\label{chap:problem_statement}

As stated by Hanson, Ruby on Rails is built to maximize the developers productivity\cite{rubyonrails}. It does so by providing a comprehensive set of tools that perform most of the work. The RoR framework is also designed for easy system evolution through \emph{migrations} --- which allow developers to easily add, remove and modify tables and table rows, while minimizing the effort of maintaining application consistency. While this is a solid, proven approach to improve an application's variability over time, it is often less flexible than one might wish, often leading to complicated data migration tasks which may involve modifying production data while ensuring consistency --- this can pose as a problem when the amount of data is extensive and highly variable in nature. As such, the use of a static database schema coupled with migrations may not be entirely desirable for applications highly variable in nature.

As presented on Chapter~\ref{chap:sota}, the usage and application of adequate design patterns is capable of mitigating some of these issues, improving both an application's variability and maintainability.

The main concerns of this thesis are how these architectural and design patterns can be effectively applied to a somewhat (architecturally speaking) restrictive framework, and how these techniques can be combined with a schema-oriented MVC architecture used by the Ruby on Rails Framework. One other point this work is concerned with is how can the variability of such systems increase and how effective they are in terms of development and performance.

\section{Architectural \& Design Patterns}\label{sec:architectural_patterns}

The evolution of architectural patterns is outside the scope of this project, mainly because the project which will serve as a base to the case studies herein described is tied to the Ruby on Rails framework and its implementation of the MVC architecture for software systems. Trying to tackle the problem by changing the underlying architecture would mean that the application would mostly have to be rebuilt from scratch, with the team having to stop development to rewrite the application and learning a whole new set of skills, which is simply not feasible. This process is usually done incrementally and opportunistically, as needs and opportunities appear. Moreover, financial issues would arise from the fact that the escolinhas.pt team would have to be trained in whichever new platform would be used.

As the modification of the underlying architecture has been discarded for the reasons stated in \ref{sec:architectural_patterns}, the next step is to try to modify only \emph{parts} of the application design, focusing on a number of different areas --- also known as hotspots --- chosen for their unique requirements in terms of variability. Having chosen these areas, it is important to define which AOM-related design patterns should be applied to solve their problems, and why.

\section{Case-study Areas}\label{sec:case-study_areas}

In order to identify the most adequate areas to focus in, a study of the design of the whole system was necessary --- this allowed the identification of the areas that had naturally occurring highly-variable requirements. For this study, three hotspots within the application were chosen.

\subsection{Roles}\label{sec:case-study_areas_roles}

In escolinhas.pt, a user is associated with a number of different roles. These roles allow to users to perform many different tasks and to have access to a number of distinct sections of the platform --- meaning they are used as the credentials to access and interact with the application. These roles are implemented using a variant of the \textsc{Organization Hierarchy} as described by Martin Fowler in \cite{fowler_accountability}. This implementation leads to a restrictive design, making the task of creating new roles or access rules unnecessarily complicated.
  
\subsection{Social network and contacts}\label{sec:case-study_areas_social_network}
Escolinhas.pt is a social platform for children aged 6 to 10. As a social platform, the network generated by the users is a big part of the application. At this time, this network is built dynamically upon request, by analyzing the relations between users. These relations are derived from the user's school, groups, and friendships. However dynamic, this method proves too restrictive, as there is no convenient and correct way to introduce exceptions in the network --- the only way would be to pollute the code with hard-coded rules. As such, the need for a more flexible system arises.

\subsection{Document Editor}\label{sec:case-study_areas_document_editor}
The document editor present in escolinhas.pt (Fig.~\ref{fig:escolinhas_pt_doc_editor}) is one of the most used components of the platform. It produces documents which, in their simplest form, have a title and a series of orderable blocks that serve as a placeholder for many different types of content (text, images, drawing, maps, etc). These documents are able to maintain a history of the modifications, in order to audit changes to its content, and to be able to publish a specific version of the document to the platform. While the editor was built with expansibility in mind, the process of creating a new type of block is not as streamlined as one may wish. On top of that, the versioning system implemented is very tightly coupled with the \textsc{ActiveRecord} (AR) implementation provided by the Ruby on Rails framework, which makes the logic for versioning unnecessarily complicated. This makes the document editor a very interesting area for this kind of study.

