\chapter*{Abstract}

Most software systems exist in an ever-changing environment. The difficulty to capture capture the requirements and craft the system to mirror these requirements often spells higher costs and development times. Moreover, these systems are not static in nature and often need to be modified to reflect changes in the environment and stakeholders' expectations.

An answer to these problems is to create information systems that are flexible enough to introduce model-level changes as quickly and cheaply as possible. One particular point of interest is to allow the end-users to modify the system definition --- which implies these modifications have to be performed without any programming knowledge or system redeploying. This can be achieved by using a meta-architecture design pattern known as Adaptive Object-Modeling (or AOM).

An Adaptive Object-Model architecture is a system architecture that represents classes, attributes, relationships, and behavior as \emph{metadata}, where the system definition is based on instances of model abstractions rather than classes: this allows for the easy modification of the domain model in runtime, discarding the need for redeploying. Also, a set of techniques have been developed in order to build the user interface automatically from the domain model, in order to create truly adaptive software.

This report is a review of the state of the art in adaptable systems. It starts by exposing the main techniques and methodologies for creating highly-variable software, analyzing the scope and objectives of each one. A problem-solving approach to this issue is then proposed, along with its respective scheduling which will be performed at Tecla Colorida, a Portuguese web development company responsible for the creation of \emph{escolinhas.pt}, an expanding Ruby on Rails application. Finally, an analysis is performed on the aforementioned related work and future options are narrowed down according to other research's results.

\chapter*{Resumo}

A grande maioria dos sistemas de informação existem em ambientes altamente variáveis. A dificuldade em capturar os requisitos de modo a modelar o sistema de acordo com estes causa custos de produção e tempos de desenvolvimento mais altos. Muitos destes sistemas não são estáticos e sofrem modificações regulares de modo a reflectir as mudanças sofridas pelo meio ambiente e expectativas dos clientes.

Uma resposta possível para este problema é a criação de sistemas de informação que sejam flexíveis o suficiente para que a introdução de modificações ao nível do modelo de domínio sejam o mais rápido e com o menor custo possível. Um ponto de interesse em particular é o de permitir que os utilizadores finais do sistema possam modificar o seu modelo de domínio --- o que implica que estas modificações sejam efectuadas sem qualquer conhecimento de programação ou da instalação do sistema em si. Este objectivo pode ser conseguido através da utilização de um padrão de desenho de meta-arquitecturas conhecido como \textit{Adaptive Object-Modeling} (ou \textit{AOM}).

Uma arquitectura baseada em \textit{Adaptive Object-Models} é uma arquitectura de sistemas que representa classes, atributos, associações e comportamentos como meta-dados, em que a definição do sistema se baseia em instâncias de abstrações do modelo em vez de classes: isto permite uma fácil modificação do modelo do sistema, removendo a necessidade de re-instalação. A par do desenvolvimento deste tipo de sistemas, uma série de técnicas e padrões foram criados para que a interface do utilizador seja passível de ser criada automaticamente.

Este relatório apresenta uma revisão bibliográfica do estado da arte em sistemas adaptativos. Inicialmente são exploradas as técnicas e metodologias mais habitualmente usadas na construção de sistemas altamente variáveis, bem como como o escopo e objectivos de cada um. Uma abordagem para a resolução do problema apresentado é então proposta, juntamente com a respectiva calendarização. Este projecto irá tomar lugar na Tecla Colorida, empresa Portuguesa de desenvolvimento web, responsável pela criação do projecto \emph{escolinhas.pt}. Por fim, é feita uma análise relativa ao trabalho mencionado anteriormente para que as escolhas futuras possam ser limitadas pelo trabalho anterior já efectuado. 

