\chapter*{Abstract}

Most software systems exist in an ever-changing environment. The difficulty to capture the requirements and craft the system to mirror these requirements often requires higher costs and development time. Moreover, these systems are not static in nature and often need to be modified to reflect changes in the environment and stakeholders' expectations.

An answer to these problems is to create information systems that are flexible enough to introduce model-level changes as quickly and cheaply as possible. One particular point of interest is to allow the end-users to modify the system definition --- which implies that modifications have to be performed without any programming knowledge or system redeploying. This can be achieved by using a meta-architecture pattern known as Adaptive Object-Modeling (or AOM).

An Adaptive Object-Model architecture is a system architecture that represents classes, attributes, relationships, and behavior as \emph{metadata}, where the system definition is based on instances of model abstractions rather than classes: this allows for the easy modification of the domain model in runtime, discarding the need for redeploying. Also, a set of techniques have been developed to build the user interface automatically from the domain model, in order to create truly adaptive software.

The platform \emph{escolinhas.pt} is an expanding Ruby on Rails application that works as a private social network for schools, students aged 6 to 10, along with their teachers and parents. It is based on the structure of real Portuguese primary education schools and promotes the use of digital tools to create collaborative work.

This report describes the work and research involved in adapting the ideologies and techniques used in building AOM architectures to already existent applications based on static database schemas and MVC architectures, such as \emph{escolinhas.pt}.

This research focuses on studying the current design for the \emph{escolinhas.pt} platform and pinpointing which modules of the application have special needs regarding their variability. Three modules or areas of the application were chosen to conduct this study: user roles (and authorization logic), the social network, and the document editor. Each one of these areas is studied in order to determine possible problems and gather the necessary variablity requirements. For each one of the areas, small, proof-of-concept prototypes are built, applying the most appropriate design patterns from a set of candidates, detailing implementation details for each one, as well as studying the impact that each prototype had.

Results show that the application of these design patterns to a MVC based framework such as Ruby on Rails are not only possible, but can have a positive impact on the overall design and performance of the application.

\chapter*{Resumo}

A grande maioria dos sistemas de informação existem em ambientes altamente variáveis. A dificuldade em capturar os requisitos de modo a modelar o sistema de acordo com estes causa custos de produção e tempos de desenvolvimento mais altos. Muitos destes sistemas não são estáticos e sofrem modificações regulares de modo a reflectir as mudanças sofridas pelo meio ambiente e expectativas dos clientes.

Uma resposta possível para este problema é a criação de sistemas de informação que sejam flexíveis o suficiente para que a introdução de modificações ao nível do modelo de domínio sejam o mais rápido e com o menor custo possível. Um ponto de interesse em particular é o de permitir que os utilizadores finais do sistema possam modificar o seu modelo de domínio --- o que implica que estas modificações sejam efectuadas sem qualquer conhecimento de programação ou da instalação do sistema em si. Este objectivo pode ser conseguido através da utilização de um padrão de desenho de meta-arquitecturas conhecido como \textit{Adaptive Object-Models} (ou AOM).

Uma arquitectura baseada em \textit{Adaptive Object-Models} é uma arquitectura de sistemas que representa classes, atributos, associações e comportamentos como meta-dados, em que a definição do sistema se baseia em instâncias de abstrações do modelo em vez de classes: isto permite uma fácil modificação do modelo do sistema, removendo a necessidade de re-instalação. A par do desenvolvimento deste tipo de sistemas, uma série de técnicas e padrões foram criados para que a interface do utilizador seja passível de ser criada automaticamente.

A plataforma \emph{escolinhas.pt} é uma applicação criada em Ruby on Rails e que funciona como uma rede social para alunos de 6 a 10 anos de idade, conjuntamente com os seus encarregados de educação e professores. A sua estrutura é baseada nas escolas primárias Portuguesas, e promove o uso de ferramentas digitais para a criação de trabalhos colaborativos.

Este relatório descreve o trabalho e pesquisa envolvido na adaptação de ideologias e técnicas usadas na construção de arquitecturas AOM a aplicações já existentes, assentes em esquemas de bases de dados estáticas e arquitecturas MVC, como é o caso da plataforma \emph{escolinhas.pt}.

Este trabalho foca-se no estudo do desenho actual da plataforma \emph{escolinhas.pt} e em capturar quais os módulos da aplicação que têm maiores necessidades de variabilidade. Três módulos da aplicação foram escolhidos para este estudo: os papéis (\emph{roles}) dos utilizadores, a rede social e o editor de documentos da plataforma. Cada uma destas áreas é estudada de modo a determinar possíveis problemas e recolher os requisitos de variabilidade de cada uma. Para cada um destes módulos pequenas provas de conceito foram construídas como protótipos, aplicando os padrões de desenho mais apropriados de um conjunto de padrões candidadatos, referindo os detalhes de implementação bem como estudando o impacto que cada uma destas soluções teve na aplicação.

Resultados mostram que a aplicação de padrões de desenho associados a arquitecturas AOM a uma framework baseada em MVC são não só possíveis, como podem ter um impacto positivo tanto no desenho como no desempenho da aplicação.
