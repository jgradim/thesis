\chapter{Approach \& Results}\label{chap:approach_results}



\section{Current Design Analysis}\label{sec:current_design_analysis}

A thorough analysis of the current design of the platform was the first step taken, in order to identify potencial problems within the platform. This study was conducted using a series of tools\footnote{Rubymine 2.0\cite{rubymine}, Rails ERD\cite{rails_erd}} that, unfortunately (at the time of writing) proved unsuccessful in extracting an \emph{Entity Relationship Diagram} from the code of the application. As such, another approach was taken: the product owner of the platform was queried on what he felt were the variability ``hotspots'' of the application: this narrowed the scope of the current design analysis, allowing focus on three different areas, as stated in~\ref{sec:case-study_areas}. These areas were then manually studied in order to extract the \emph{Entity Relationship Diagrams} relative to each one, which \textbf{\underline{allowed}} analyzing each one as indenpendently as possible, so that the applied design patterns (if any) would emerge --- making the task of pinponting exactly what was wrong with each approach much easier.

\section{Variability Analysis}\label{sec:variability_analysis}

