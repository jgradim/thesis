\chapter{Technologies}\label{chap:technologies}

\section{C\#}\label{sec:csharp}

C\# (C-Sharp) is a multi-paradigm, strongly-typed programming language developed by Microsoft within the .NET framework. C\# is an object-oriented language, but includes support for other programming paradigms such as functional, imperative, and component-oriented. It was built to be a robust and performant language, featuring strong type-checking, array bounds checking, detection of attempts to use uninitialized variables, and automatic garbage collection~\cite{csharp}.

\section{Oghma}\label{sec:oghma}

Oghma is meant to be a reference framework for the development of AOM systems, developed in C\#. It allows the rapid creation of highly variable, dynamic systems. It is currently being developed in the context of the doctoral thesis of Hugo Sereno Ferreira. It is a very complete framework able to create, manage and persist AOM systems, from backend to GUI generation.

Oghma is thus a concrete implementation of a framework based on the reference architecture to develop AOM-based systems established in~\cite{ferreira_phd_2010}, that balances several design and engineering forces. It supports the creation of models resembling MOF~\cite{mof} and UML, and aims at covering the entire cycle of system creation and evolution. As an AOM, it allows the introduction of changes to the system during runtime, thus providing a particular kind of confined end-user development.

Furthermore, the framework leverages the infrastructure used to support system evolution to provide additional features, such as auditing over the system’s usage, and time-traveling to an arbitrary point along its evolution (i.e. to set the system in a past state).

Oghma includes a set of interchangeable components designed to have an high degree of flexibility, as it was designed to support several types of persistency engines --- be it in memory or a DBMS.

\section{Ruby}\label{sec:technologies:ruby}

Ruby is a dynamic, purely object oriented programming language, developed by Yukihiro Matsumoto, released in 1995. Ruby supports multiple programming paradigms, including functional, object-oriented, imperative and reflexive. It was inspired by many different languages such as Lisp, Smalltalk, Perl and Ada, and possesses a series of characteristics that make it extremely attractive~\cite{ruby}.

