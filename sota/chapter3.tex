\chapter{Capítulo Exemplo}\label{chap:chap3}

\section*{}

Neste capítulo apresentam-se exemplos de formatação de figuras e
tabelas, equações e referências cruzadas.

Maecenas eleifend facilisis leo. Vestibulum et
mi. Aliquam posuere, ante non tristique consectetuer, dui elit
scelerisque augue, eu vehicula nibh nisi ac est. 
Suspendisse elementum sodales felis. Nullam laoreet fermentum urna. 

\section{Introdução}

Apresenta-se de seguida um exemplo de equação, completamente fora do contexto:
\begin{eqnarray}
CIF_1: \hspace*{5mm}F_0^j(a) &=& \frac{1}{2\pi \iota} \oint_{\gamma} \frac{F_0^j(z)}{z - a} dz\\
CIF_2: \hspace*{5mm}F_1^j(a) &=& \frac{1}{2\pi \iota} \oint_{\gamma} \frac{F_0^j(x)}{x - a} dx \label{eq:cif}
\end{eqnarray}

Na Equação~\ref{eq:cif} lorem ipsum dolor sit amet, consectetuer
adipiscing elit. Suspendisse tincidunt viverra elit. Donec tempus
vulputate mauris. Donec arcu. Vestibulum condimentum porta
justo. Curabitur ornare tincidunt lacus. Curabitur ac massa vel ante
tincidunt placerat. Cras vehicula semper elit. Curabitur gravida, est
a elementum suscipit, est eros ullamcorper quam, sed cursus velit
velit tempor neque. Duis tempor condimentum ante. Nam
sollicitudin. Vestibulum adipiscing, orci eu tempor dapibus, risus
sapien porta metus, et cursus leo metus eget nibh. 

Pellentesque rutrum, sapien at viverra facilisis, metus eros blandit
sem, quis dictum erat metus eget erat. Vivamus malesuada dapibus
nulla. Maecenas nec purus. Suspendisse auctor mattis augue. Phasellus
enim nisi, iaculis sit amet, pellentesque a, iaculis in, dui. Integer
risus. 

\section{Secção Exemplo}

A arquitectura do visualizador assenta sobre os seguintes conceitos
base~\citep{kn:ZPMD97}: 

\begin{itemize}
\item \textbf{Componentes} --- Um sistema é definido por um conjunto
  de componentes; a colaboração é baseada no modelo \emph{push}.
\item \textbf{Praesent} --- Sit amet sem maecenas eleifend facilisis leo.
\item \textbf{Pellentesque} --- Habitant morbi tristique senectus et netus et
\end{itemize}

\subsection{Exemplo de Figura}

É apresentado na Figura~\ref{fig:arch} da página~\pageref{fig:arch} um
exemplo de figura flutuante que deverá ficar no topo da página.

\begin{figure}[t]
  \begin{center}
    \leavevmode
    \includegraphics[width=0.86\textwidth]{puzzle}
    \caption{Arquitectura da Solução Proposta}
    \label{fig:arch}
  \end{center}
\end{figure}

Loren ipsum dolor sit amet, consectetuer adipiscing elit. 
Praesent sit amet sem. Maecenas eleifend facilisis leo. Vestibulum et
mi. Aliquam posuere, ante non tristique consectetuer, dui elit
scelerisque augue, eu vehicula nibh nisi ac est. Suspendisse elementum
sodales felis. Nullam laoreet fermentum urna. 

Duis eget diam. In est justo, tristique in, lacinia vel, feugiat eget,
quam. Pellentesque habitant morbi tristique senectus et netus et
malesuada fames ac turpis egestas. Fusce feugiat, elit ac placerat
fermentum, augue nisl ultricies eros, id fringilla enim sapien eu
felis. Vestibulum ante ipsum primis in faucibus orci luctus et
ultrices posuere cubilia Curae; Sed dolor mi, porttitor quis,
condimentum sed, luctus in. 

\subsection{Exemplo de Tabela}

É apresentado na Tabela~\ref{tab:exemplo} um exemplo de tabela
flutuante que deverá ficar no topo da página.

\begin{table}[t]
  \centering
  \caption{Tabela Exemplo}
\begin{tabular}{|c|r@{.}lr@{.}lr@{.}l||r|}
	\hline
\multicolumn{8}{|c|}
	{\rule[-3mm]{0mm}{8mm}Iteração $k$ de $f(x_n)$} \\
\textbf{\em k}
	& \multicolumn{2}{c}{$x_1^k$}
	& \multicolumn{2}{c}{$x_2^k$}
	& \multicolumn{2}{c||}{$x_3^k$}
	& comentários \\ \hline \hline
0   & -0&3                 & 0&6                 &  0&7   & - \\
1   &  0&47102965 & 0&04883157 & -0&53345964  & $\delta<\epsilon$ \\
2   &  0&49988691 & 0&00228830 & -0&52246185  & $\delta < \varepsilon$ \\
3   &  0&49999976 & 0&00005380 & -0&523656   &   $N$ \\
4   &  0&5                 & 0&00000307 & -0&52359743  & \\
\vdots	& \multicolumn{2}{c}{\vdots}
	& \multicolumn{2}{c}{$\ddots$}
	& \multicolumn{2}{c||}{\vdots}  & \\
7   &  0&5   & 0&0    & \textbf{-0}&\textbf{52359878}
		 & $\delta<10^{-8}$ \\ \hline
\end{tabular}
  \label{tab:exemplo}
\end{table}

Loren ipsum dolor sit amet, consectetuer adipiscing elit. 
Praesent sit amet sem. Maecenas eleifend facilisis leo. Vestibulum et
mi. Aliquam posuere, ante non tristique consectetuer, dui elit
scelerisque augue, eu vehicula nibh nisi ac est. Suspendisse elementum
sodales felis. Nullam laoreet fermentum urna. 

Duis eget diam. In est justo, tristique in, lacinia vel, feugiat eget,
quam. Pellentesque habitant morbi tristique senectus et netus et
malesuada fames ac turpis egestas. Fusce feugiat, elit ac placerat
fermentum, augue nisl ultricies eros, id fringilla enim sapien eu
felis. Vestibulum ante ipsum primis in faucibus orci luctus et
ultrices posuere cubilia Curae; Sed dolor mi, porttitor quis,
condimentum sed, luctus in. 

\section{Secção Exemplo}

Loren ipsum dolor sit amet, consectetuer adipiscing elit. 
Praesent sit amet sem. Maecenas eleifend facilisis leo. Vestibulum et
mi. Aliquam posuere, ante non tristique consectetuer, dui elit
scelerisque augue, eu vehicula nibh nisi ac est. Suspendisse elementum
sodales felis. Nullam laoreet fermentum urna. 

Duis eget diam. In est justo, tristique in, lacinia vel, feugiat eget,
quam. Pellentesque habitant morbi tristique senectus et netus et
malesuada fames ac turpis egestas. Fusce feugiat, elit ac placerat
fermentum, augue nisl ultricies eros, id fringilla enim sapien eu
felis. Vestibulum ante ipsum primis in faucibus orci luctus et
ultrices posuere cubilia Curae; Sed dolor mi, porttitor quis,
condimentum sed, luctus in. 

\section{Resumo}

Pellentesque habitant morbi tristique senectus et netus et
malesuada fames ac turpis egestas. Fusce feugiat, elit ac placerat
fermentum, augue nisl ultricies eros, id fringilla enim sapien eu
felis. Vestibulum ante ipsum primis in faucibus orci luctus et
ultrices posuere cubilia Curae; Sed dolor mi, porttitor quis,
condimentum sed, luctus in. 
