\chapter{Introduction}\label{chap:intro}

\section{Context}\label{sec:context}

Software systems are usually designed with a specific purpose in mind\cite{}. They rely on a series of requirements which are often very difficult to capture and maintain, as they have a tendency to evolve faster than the implementation. This is caused mainly by the poor understanding by the stakeholders about their needs and expectations about what a software system should be able to do\cite{PT07}. These situations lead to higher costs in software development, as creating and maintaining software systems is a knowledge intensive task\cite{AdOdSBD07}. Moreover, most of these system are not static, and have a constant need to evolve in order to adapt themselves to their environment and new business rules, shifting the stakeholders' needs and expectations about these software systems.

In face of these situations, new development methods started to focus more on iterative and incremental approaches, accepting \textit{incompleteness} as part of every software system development cycle\cite{WC03}. At the same time, many new systems are being developed with an emphasizis on flexibility and run-time configuration\cite{YJ02}. These approaches present a clear contrast with an up-front, full specification for a software system, which, albeit benefitial for some cases, are impratical in constant evolution scenarios. This leads to a change in software requirements and inevitable refactoring, which in turn may lead to a \textsc{big ball of mud}, ultimately leading to unmaintainable systems, very costly to modify and adapt to the stakeholders needs\cite{BIGBALLOFMUD}.

\begin{quote}
  ``Like the Abstract, the Introduction should be written to engage the
  interest of the reader. It should also give the reader an idea of
  how the dissertation is structured, and in doing so, define the
  thread of the contents.''~\citep[chap.\ Introduction]{kn:Tha01}
\end{quote}

\section{Motivation and Objectives}\label{sec:goals}



\section{Report Overview}\label{sec:structure}

