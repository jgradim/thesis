\chapter{State Of The Art in Adaptive Systems}\label{chap:sota}

\section{Generative Programming}\label{sec:generative_programming}

Generative programming methods approach the adaptability of a system by using an ontological model representative of this system to automatically create executable artifacts or code skeleton than can be further refined according to different needs.

Rails scaffolding is an excelent example of this approach, which will be further explained in Section \ref{sec:ror}

\subsection{Model-Driven Architectures}\label{sec:mda}

Model-Driven Architectures derives a system's functionality from a platform-independent model (PIM) by using a domain specific language (DSL). This representation is then mapped to a platform-specific model

\subsection{Ruby On Rails}\label{sec:ror}

\subsection{Frameworks}\label{sec:frameworks}

%-------------------------------------------------------------------------------

\section{Meta-Architecture}\label{sec:meta-architecture}

\subsection{Meta Programming}\label{sec:metaprogramming}

\subsection{Smalltalk}\label{sec:smalltalk}

\subsection{Ruby}\label{sec:ruby}

\subsection{Adaptive Object-Models}\label{sec:aom}

