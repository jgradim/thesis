\chapter{State Of The Art in Adaptive Systems}\label{chap:sota}

\section{Generative Programming}\label{sec:generative_programming}

Generative programming methods approach the adaptability of a system by using an ontological model representative of this system to automatically create executable artifacts or code skeleton than can be further refined according to different needs.

Rails scaffolding is an excelent example of this approach, which will be further explained in Section \ref{sec:ror}

\subsection{Model-Driven Engineering}\label{sec:mda}

Model-Driven Engineering first appeared in 2001 \cite{mda_omg} as an answer to the growing complexity system architectures. This growing complexity and the lack of a integrated view ``forced many developers to implement suboptimal solutions that unnecessarily duplicate code, violate key architectural principles, and complicate system evolution and quality assurance'' \cite{Sch06}.

\subsection{Ruby On Rails}\label{sec:ror}

\subsection{Frameworks}\label{sec:frameworks}

%-------------------------------------------------------------------------------

\section{Meta-Architecture}\label{sec:meta-architecture}

\subsection{Meta Programming}\label{sec:metaprogramming}

A metaprogramming system is a programming facility (subprogramming system or language) whose basic data objects include the programs and program fragments of some particular programming language, known as the target language of the system. Such systems are designed to facilitate the writing of metaprograms, that is, programs about programs. Metaprograms take as input programs and fragments in the target language, perform various operations on them, and possibly, generate modified target-language programs as output \cite{CI84}. % COPIADO DIRECTAMENTE DO PAPER!

\subsection{Smalltalk}\label{sec:smalltalk}

\subsection{Ruby}\label{sec:ruby}

\subsection{Adaptive Object-Models}\label{sec:aom}

