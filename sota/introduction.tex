\chapter{Introduction}\label{chap:intro}

\section{Context}\label{sec:context}

Software systems are usually designed with a specific purpose in mind. They rely on a series of requirements which are often very difficult to capture and maintain, as they have a tendency to evolve faster than the implementation. This is caused mainly by the poor understanding by the stakeholders about their needs and expectations about what a software system should be able to do~\cite{PT07}. These situations lead to higher costs in software development, as creating and maintaining software systems is a knowledge intensive task~\cite{AdOdSBD07}. Moreover, most of these system are not static, and have a constant need to evolve in order to adapt themselves to their environment and new business rules, shifting the stakeholders' needs and expectations about these software systems.

In face of these situations, new development methods started to focus more on iterative and incremental approaches, accepting \textit{incompleteness} as part of every software system development cycle~\cite{WC03}. At the same time, many new systems are being developed with an emphasizis on flexibility and run-time configuration~\cite{YJ02}. These approaches present a clear contrast with an up-front, full specification for a software system, which, albeit benefitial for some cases, are impratical in constant evolution scenarios. This leads to a change in software requirements and inevitable refactoring, which in turn may lead to a \textsc{big ball of mud}, ultimately leading to unmaintainable systems, very costly to modify and adapt to the stakeholders needs~\cite{FY97}.

\section{Motivation and Objectives}\label{sec:goals}

Developing quality software is a costly process. Maintaining, as well as adding new features to these software systems is a costly and time-consuming process. If the end-user is allowed to make changes to an information system in runtime, it can effectively shape the system's architecture (model and relation-wise) to suit his or her needs according to the natural evolution of the business rules, enviroment changes and needs.

The main goal of this project is to take the Oghma framework~\cite{FCA09} to a web context, by implementing a web browser compatible graphical user interface, akin to the GUIs generated automatically by the framework itself. A direct interface with Ruby On Rails will also be created in order to take advantage of all the facilities provided by this framework. As a sub-product of this work, a proof-of-concept customizable quiz application will be developed that will allow teachers to create tests adapted to each of the student's needs, giving them full control over the concepts present in each of the developed models.

This project will be applied to the \textit{Escolinhas}~\cite{escolinhas}, a growing, Portuguese, Ruby On Rails based project. Escolinhas aims at sustaining social and collaborative work for children in elementary schools involving students, teachers and parents as its users. With the teachers demanding better and more adaptable teaching tools, it becomes an excellent case-study application to research, test and apply all the work and discoveries made during the course of this thesis.

% Establish a reference framework using the concepts of web 2.0 for AOM systems
% Understand GUI patterns that allow end-users to manipulate domain models
% Validate through an industrial use-case application

\section{Report Overview}\label{sec:structure}

The rest of this report is structured as follows:\\

%\textbf{Chapter \ref{chap:technologies}: ``\nameref{chap:technologies}'' } presents the main technologies which will be part of the development process.\\

\textbf{Chapter \ref{chap:sota}: ``\nameref{chap:sota}'' } reviews the most important methodologies and patterns used to make software as adaptable and maintable as possible.\\

\textbf{Chapter \ref{chap:approach}: ``\nameref{chap:approach}'' } suggests a possible approach to solving the aforementioned problem, as well as the work plan for the thesis.\\

\textbf{Chapter \ref{chap:conclusions}: ``\nameref{chap:conclusions}'' } tries to conclude the most successful path of development by using related work’s conclusions, exposed in the state of the art. % para ir na direcção correcta

